\documentclass[10pt,xcolor=x11names,compress, notes=show]{beamer}% pour l'impression, tout n'apparait qu'une fois \documentclass[handout,12pt]{beamer}

%\documentclass[xcolor=x11names]{beamer}
%\usepackage[scaled]{helvet}
\usepackage[round]{natbib}



\usepackage[utf8x]{inputenc}
\usepackage{ucs}
\usepackage[french]{babel}
\usepackage{todonotes}
\usepackage{tikz}
\usepackage{color}
%\usepackage{subfigure}
%\usepackage[]{geometry}
\usepackage{changepage}
\usetikzlibrary{calc}

\usepackage{bm}
\usepackage{pifont} %pour les symbole sympa \ding{nb}
\usepackage[export]{adjustbox}
\usepackage{subcaption}


\setbeamertemplate{navigation symbols}{} 


%\usepackage{palatino}

%pour le theme
%\usetheme{CambridgeUS}

%\usetheme{Goettingen}
\useinnertheme{default}
\useoutertheme[subsection=false]{miniframes}
\setbeamertemplate{blocks}[rounded][shadow=true]
\setbeamercolor{block title}{fg=DeepSkyBlue4,bg=DeepSkyBlue4!10}
\setbeamercolor{block title alerted}{bg=DeepSkyBlue4!0} 
\setbeamercolor{block title example}{bg=DeepSkyBlue4!20}
\setbeamercolor*{lower separation line head}{bg=DeepSkyBlue4} 

\setbeamerfont{title like}{shape=\scshape}
\setbeamercolor{frametitle}{fg=DeepSkyBlue4}
\setbeamercolor{title}{fg=DeepSkyBlue4}
\setbeamercolor{itemize item}{fg=black}
\setbeamercolor{itemize subitem}{fg=black}
\setbeamercolor{toc}{fg=DeepSkyBlue4}
\usepackage{amsmath,mathtools}
\usefonttheme[onlymath]{serif}

\setbeamertemplate{frametitle}{\vspace{0.13cm}\hspace{-0.9cm} \insertframetitle}
\setbeamerfont{frametitle}{size=\Large}

\setlength{\fboxrule}{1pt}

\setbeamertemplate{bibliography item}[]

%couleur table des matières
\usepackage{hyperref}
\hypersetup{colorlinks=true, linkcolor=DeepSkyBlue4}

\setbeamertemplate{caption}{\raggedright\insertcaption\par}

%Mettre la section courante en titre de diapo (pour champ de titre non-vide)
%\addtobeamertemplate{frametitle}{\frametitle{\insertsubsectionhead}}{}

\addtobeamertemplate{footline}{\hspace{11cm} \insertframenumber/\inserttotalframenumber}

\newcommand{\tr}[1]{\prescript{t\hspace{-0.08cm}}{}{#1}}



%page de titre
\author{Alice \textsc{Dinsenmeyer} \\~\\ encadrée par\\ Romain \textsc{Brossier} \& Ludovic \textsc{Moreau} \\ Maîtres de conférences, ISTerre}

\title{Imagerie ultrasonore par inversion de formes d'onde}
\subtitle{}
\date{\small 12 juillet 2016}

\titlegraphic{
\begin{minipage}{0.3\textwidth}
	\centering
	\includegraphics[height=1cm]{img/univ.png}	
\end{minipage}
\begin{minipage}{0.3\textwidth}
	\centering	
	\includegraphics[height=1cm]{img/cnrs.png}
\end{minipage}
\begin{minipage}{0.3\textwidth}
	\centering
	\includegraphics[height=1cm]{img/isterre.png}
\end{minipage}

}


\begin{document}

\begin{frame}
	\titlepage 
\end{frame}



\subsection*{}
\begin{frame}{\insertsectionhead}
	\begin{itemize}
		\item<1-> Fonction de coût : $C(\bm{m})=\frac{1}{2}||\bm{d}_{obs}-\bm{d}_{cal}(\bm{m})||^{2}$ \\[0.5cm]
		\item \only<1-1>{Perturbation du modèle : $\bm{\Delta m}=-(C'')^{-1}$\fcolorbox{white!0}{white!0}{$C'$}} \onslide<2->{Perturbation du modèle : $\bm{\Delta m}=-(C'')^{-1}$\fcolorbox{DeepSkyBlue4}{white!0}{$C'$}}
	\end{itemize}
	\onslide<2-> {
		\begin{figure}
			\vspace{-0.84cm}\hspace{2.8cm}\begin{tikzpicture}	
					\node (grad) at (10,0) {};
					\node (exp) at ($(grad)+(-1.5,-1.1)$) {};
					\draw[->, thick,shorten <=2pt,shorten >=2pt,DeepSkyBlue4] (grad) -- (exp);		
				\end{tikzpicture}
		\end{figure}
	
	
%	\only<2-2>{
%		\vspace{-0.8cm}\begin{equation}
%			\frac{\partial C}{\partial m_{i}}= \tr{\tilde{\bm{d}}_{cal}} \tr{\left( \frac{\partial\bm{A}}{\partial m_{i}} \right)}\bm{A}^{-1} (\tilde{\bm{d}}_{obs} - \tilde{\bm{d}}_{cal}) 
%		\end{equation}
%	}

		\vspace{-0.8cm}\begin{equation}
			\frac{\partial C}{\partial m_{i}}= \tr{\tilde{\bm{d}}_{cal}} \tr{\left( \frac{\partial\bm{A}}{\partial m_{i}} \right)} \textcolor{DeepSkyBlue4}{\underbrace{\textcolor{black}{\bm{A}^{-1} (\tilde{\bm{d}}_{obs} - \tilde{\bm{d}}_{cal})}}_{\text{résidus rétropopagés}}}
		\end{equation}

			~\\$\bm{A}$ : opérateur équation d'onde (élastique ou acoustique)
	}
\end{frame}


\begin{frame}[allowframebreaks]{Références}

	\begin{adjustwidth}{-2em}{-2.5em}
		\scriptsize
		\bibliographystyle{abbrvnat}
		\bibliography{biblio}
	\end{adjustwidth}
\end{frame}

\begin{frame}
	questions : différence avec tomo diffraction\\
	défaut : air
\end{frame}


\end{document}

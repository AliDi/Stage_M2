\chapter{L'inversion de formes d'onde \label{fwi}}
par opposition à des inversion type tomo qui n'utilisent que partiellement les infos, fwi ne fait pas d'hypothèse sur le champ d'onde. s


\section{ce que c'est}
"inversion approach resembles prestack, reverse-time mi-
gration but differs in that the problem is formulated in
terms of velocity (not reflectivity), and the method is
fully iterative."PRATT99

rodirguez 2014 : Full waveform
inversion should not to be mistaken for migration techniques that
are based on Claerbout’s ‘‘imaging principle’’ : J.F. Claerbout, Toward a unified theory of reflector mapping, Geophysics 36 (3)
(1971) 467–481.which defines a
reflectivity field by the ratio of upgoing and downgoing wave
fields. Neverthelesss

\section{Problème direct}
formulation forte ou faible : 
FDTD, FEM,...

\section{le pb inverse}

\section{Optimisation}

brossier : explication du choix de la norme l2


\section{resultats en geophy}

\section{application au cnd de soudure : les problématiques}

-guide d'onde
-acquisition en surface seulement, et problématique de la soudure bombée
-anisotropie (cf image soudure) forte, qui touche not. les ondes S.
-acquisition horizontale pas idéale pour inverser la vitesse horizontale (car petits offsets et peu de courbure de rayon comme en géophys) (discuter le choix des paramètres à inverser compte-tenu de la configuration)
-sources et récepteurs mobiles 
-geophysique, dispositif de surface, donc on ne considère que les diffractions rayonnant vers la surface (soit angle de diffraction de max 180°)(Forgues, pages 160). En CND, on illumine des deux côté


\begin{figure}
	\includegraphics[height=5cm]{./img/soudure1.png}
	\includegraphics[height=5cm]{./img/soudure2.png}
	\caption{Macrographie d'une soudure industrielle en acier inoxydable en acier austénitique \citep{chassignole}. À gauche : coupe dans le plan $(x,z)$, à droite : coupe dans le plan $(x,y)$.}
\end{figure}
\todo[inline]{préciser les plans sur un schéma et l'orientation des photos}

grains colonnaires

p91 potel bruneau : données "d'aspect limité" : il n'es tpas possible de tourner autour de l'obstace. On colpense la perte d'info en réalisant les mesures sur plusieurs freq et possibilité de déplacer capteur.



\addcontentsline{toc}{section}{\textbf{Introduction}}
\chapter*{Introduction}

Dans l'industrie ou le domaine médical, il est nécessaire de connaître les propriétés et contrôler l'évolution de matériaux élastiques. En particulier, le contrôle non-destructif (CND) de soudures a des enjeux majeurs en terme de sécurité. Une bonne qualité d'image est indispensable pour contrôler l'évolution des soudures de structures critiques, telles que les systèmes de refroidissement de centrale nucléaire ou les canalisations pour le transport de matières fluides (hydrocarbure, gaz, produits chimiques,...) .  Pour cela, en complément des ondes électromagnétiques,  les ondes ultrasonores sont utilisées en raison de leur plus faible atténuation.\\~\\ \\

La forte anisotropie induite par la cristallisation du métal dans les soudures austénitiques est peu prévisible et courbe le faisceau ultrasonore. La structure granulaire des soudures et les phénomènes de propagation associés sont de mieux en mieux connus \citep{moysan}, mais les paramètres de soudage sont très variés \citep{chassignole} et parfois aléatoires. Or, les méthodes actuelles d'imagerie de soudures nécessitent une bonne connaissance \emph{a priori} de la vitesse de propagation des ondes élastiques pour interpréter correctement les temps de vol des ondes mesurées.  Cette vitesse étant mal connue, il est difficile de localiser et dimensionner précisément les défauts recherchés (porosité, fissure, manque de fusion, corrosion, corps étrangers, ...).\\ 


Ce stage a pour but de tester et d'adapter, à l'imagerie de soudure par ultrasons, une méthode qui reconstruit directement les paramètres élastiques :  l'inversion complète de formes d'onde (Full Waveform Inversion en anglais, notée FWI). La FWI est une méthode d'imagerie quantitative haute résolution, principalement développée dans un contexte de prospection géophysique depuis plusieurs décennies \citep{tarantola_84}. Elle est basée sur la résolution d'un problème d'optimisation et permet une reconstruction des paramètres élastiques en tout point d'un milieu discrétisé. Des démarches similaires existent déjà dans les domaines de l'imagerie médicale \citep{devaney_82} et du contrôle non destructif \citep{dominguez}. \\

Après une présentation des méthodes actuelles d'imagerie, le principe de la FWI est développé d'une part, ainsi que les problématiques qui lui sont liées d'autre part.
Une troisième partie est dédiée aux applications de la FWI en domaine temporel à des données simulées à partir de soudures simplifiées par une hypothèse de propagation acoustique à deux dimensions. Une étude de résolution de la FWI est proposée et quelques résultats d'inversions en milieux isotrope et isotrope transverse sont discutés.









%\todo[inline]{méthodes vieilles et actuelles (principe et résultats, avantages et limites)}
%Différentes méthodes de reconstruction d'image à partir de données de mesures peuvent être utilisées.\\
%
%Les échographies obtenues à l'aide de transducteurs mono-éléments représentent directement les échos en un point (A-Scan), sur une coupe (B-Scan) ou une tranche (C-Scan) du matériau (image type annexe k thèse chassignol ?). L'obtention d'une image 2D nécessite donc un balayage sur l'ensemble d'une surface de la pièce à contrôler. \\
%
%D'autre méthode d'imagerie utilisant des transducteurs multi-éléments sont également basées sur les temps d'arrivée des ondes.  tfm, saft; tofd\\
%résolution limitée par le nombre d'éléments et leur espacement
%
%
%Pour toutes ces méthodes, il est important de connaître la vitesse du milieu observé, afin de localiser le réflecteur à l'origine de l'écho observé. C'est pour cette raison que leur efficacité est fortement réduite en milieux anisotrope.  
%De plus, la résolution de ces méthodes est limitée par la capacité à séparer les échos et est donc de l'ordre de la longueur d'onde du signal d'excitation. \\
%
%
%Dépassant ces limitations, d'autres méthodes de formation de voies dites "haute résolution" utilisent les techniques de retournement temporel telles que la méthode de Décomposition de l'Opérateur de Retournement Temporel \citep{prada_dort} ou la méthode Capon \citep{capon} dont est issu notamment l'algorithme MUltiple SIgnal Classification \citep{schmidt}. Ces méthodes sont basées sur la décomposition en valeurs propres de la matrice interspectrale et permettent ainsi de minimiser la contribution énergétique du bruit. Cependant, ces méthodes restent sensibles au bruit. De plus, tout comme les méthodes de formation de voies classiques, elles nécessitent de connaître a priori les propriétés élastiques du milieu de propagation des ondes.
%
%
%
%
%
%\todo[inline]{points communs avec imagerie de la terre}
%Le domaine de la prospection géophysique utilise également les ondes mécaniques pour établir les propriétés de la Terre. Si le matériel et les gammes de  fréquence diffèrent, la physique et les méthodes sont proches
%
%
%tomographie (basée sur rais, ie hf, pas de réfraction, isotrope), migration (sensible au bruit, ondes réfléchies seulement, , optimisation topologique (? Tarantola ?)
%
%\todo[inline]{fwi, son contexte}
%topological optimization
%oberai
%
%\todo[inline]{présentation travail de stage et plan du rapport}


	


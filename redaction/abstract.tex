\section*{Abstract}

Weld imaging is crucial to control health of cooling systems and pipeline. But methods currently used do not take well into account the strong anisotropy induced by grain structure of welds. The time shift caused by anisotropy cannot be predicted and thus, defects in weld cannot be localised precisely. \\

The full waveform inversion (FWI) attempts to build an image of elastic parameters and hence could take accurately into account anisotropic propagation. It is based on an optimisation problem that aims to reduce the misfit between recorded and computed data by perturbing the model parameters. \\

A study of resolution shows that FWI can provide high resolution image depending on the acquisition system, the records duration and the scattering pattern of the different parameters. Some applications of time-domain FWI to 2D-acoustic welds are performed under isotropic and vertical isotropic propagation approximation. It appears that surface acquisition makes horizontal velocity hard to build, even with wide azimuthal data. Moreover, multiparametric FWI, which is challenging because it increases ill-posedness of the inversion, does not improve significantly the image quality, compared to vertical velocity monoparametric inversion.


%\smallskip \textbf{Mot-clés} : ECND par ultrasons, problèmes d'optimisation,  
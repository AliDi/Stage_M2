\chapter{Application de la FWI à des données simulées}

p91 potel bruneau en francais: données "d'aspect limité" : il n'est pas possible de tourner autour de l'obstacle. On compense la perte d'info en réalisant les mesures sur plusieurs freq et possibilité de déplacer capteur.

\section{problématiques des inversions multiparamètres}



\section{Résolution}
En théorie, si l'éclairage est parfait, on est limité en résolution par lambda/2 (cf review virieux). En pratique, tout comme en ray-tomo (cf wiliamson cité dans review virieux), on est très limité par l'éclairage.

image du remplissage en nb d'onde -> dépendance des sources et du nombre de réflexion surfaces libres.

\section{anisotrope}

anisotrope est plus problématique que isotrope car : 
-modélisation plus complexe,
-problème moins bien posé

Gholami 2011 : la vitesse a beaucoup plus d'inflence sur les données que les paramètre delta et epsilon (delta étant le plus faible). D'après ses schémas, on va donc avoir une maj de la vitesse mais pas des autres paramètres

\subsection{VTI}
Pose quelques problèmes (Duveneck 2008) notamment génération d'onde S (sur données "vrai simulée"  et sur problème direct, mais pas la même car différente grille, PML, ... donc on la mute sur le résidu) qui n'a pas de sens physique. Proposer les solutions (taper Epsilon, en sismo on est dans l'eau donc c'est fait naturellement).


Solution : sources/recep qui apportent le plus d'info. Aller voir Jean à ce sujet


VTI elliptique ne gène pas l'inversion car beaucoup d'info portées par la transmission (vecteurs d'ondes verticaux non affectés par l'anisotropie elliptique VTI)  -> pas assez proche du modèle réel

le flop du vti montre que l'inversion des paramètres est tributaire de l'acquisition

Passer en non-elliptique, avec delta qui donne la near vertical anisotropy. Romain : passer en TTI (code en freq)

